\documentclass[a4paper, 11pt, oneside, polutonikogreek, french]{article}
\usepackage[T1]{fontenc}
\usepackage{aurical}

% Load encoding definitions (after font package)

\usepackage{textalpha}

\usepackage{listings}
\lstset{basicstyle=\ttfamily}

% Babel package:
\usepackage[french]{babel}

% With XeTeX$\$LuaTeX, load fontspec after babel to use Unicode
% fonts for Latin script and LGR for Greek:
\ifdefined\luatexversion \usepackage{fontspec}\fi
\ifdefined\XeTeXrevision \usepackage{fontspec}\fi

% "Lipsiakos" italic font `cbleipzig`:
\newcommand*{\lishape}{\fontencoding{LGR}\fontfamily{cmr}%
		       \fontshape{li}\selectfont}
\DeclareTextFontCommand{\textli}{\lishape}

\usepackage{booktabs}
\setlength{\emergencystretch}{15pt}
\usepackage{fancyhdr}
\usepackage{microtype}
\begin{document}
\Fontauri

\renewcommand\thefootnote{\Fontauri{\arabic{footnote}}}

\begin{titlepage} % Suppresses headers and footers on the title page
	\centering % Centre everything on the title page
	%\scshape % Use small caps for all text on the title page

	%------------------------------------------------
	%	Title
	%------------------------------------------------
	
	\rule{\textwidth}{1.6pt}\vspace*{-\baselineskip}\vspace*{2pt} % Thick horizontal rule
	\rule{\textwidth}{0.4pt} % Thin horizontal rule
	
	\vspace{1\baselineskip} % Whitespace above the title
	
	{\scshape\Huge Dissertation sur les Bætyles}
	
	\vspace{1\baselineskip} % Whitespace above the title

	\rule{\textwidth}{0.4pt}\vspace*{-\baselineskip}\vspace{3.2pt} % Thin horizontal rule
	\rule{\textwidth}{1.6pt} % Thick horizontal rule
	
	\vspace{1\baselineskip} % Whitespace after the title block
	
	%------------------------------------------------
	%	Subtitle
	%------------------------------------------------
	
	{\scshape \Large Par M. Falconet} % Subtitle or further description
	
	\vspace*{1\baselineskip} % Whitespace under the subtitle
	
        {\scshape\scriptsize } % Subtitle or further description
    
	%------------------------------------------------
	%	Editor(s)
	%------------------------------------------------
        \vspace*{\fill}

	\vspace{1\baselineskip}

	{\small\scshape Paris 1722}
	
	{\small\scshape{De L'Imprimerie Royal}}
	
	\vspace{0.5\baselineskip} % Whitespace after the title block

        \scshape Internet Archive Online Edition  % Publication year
	
	{\scshape\small Utilisation non commerciale --- Partage dans les mêmes conditions 4.0 International} % Publisher
\end{titlepage}
\setlength{\parskip}{1mm plus1mm minus1mm}
\clearpage
\Large
\pagestyle{fancy}
\fancyhf{}
\cfoot{\Fontauri{\thepage}}
\paragraph{}
Dans le Poème des Pierres attribué à Orphée, immédiatement après l'article de l'Aimant,\footnote{\Fontauri{A l'article intitulé Ὀστρίτνς.}} on lit la description d'une Pierre appelée \emph{Ophitès}, et d'un nom plus particulier, le vrai \emph{Sideritès}. Quelque fabuleux que soit ce que le Poète dit de l'Aimant, aussi bien que des autres Pierres, il se surpasse lui-même dans ce qu'il nous raconte du Sideritès. Les propriétés extravagantes qu'il lui attribué, m'ont fait ressouvenir de ce que j'avois remarqué sur les Pierres appelées \emph{Bætyles} et j'ai cru les reconnaître ici sous un nom différent : cela m'a donné envie de relire ce que j'avois autrefois recueilli sur les \emph{Bætyles}, et de vous donner le précis de ce recueil en y ajoutant quelques réflexions. Le sujet est peut-être encore plus curieux qu'il ne le paraît aux Savants à qui il est déjà connu ; mais ce qui me détermine le plus à vous entretenir des merveilles des \emph{Bætyles}, c'est que je crois avoir trouvé dans les opinions de ceux qui ont mal entendu l'Histoire Naturelle, l'origine d'un merveilleux qui semble d'abord n'avoir d'autre fondement, que la seule bizarrerie de l'esprit humain livré à la superstition.

Voici d'abord ce que\footnote{\Fontauri{\emph{Ibid.}}} dit le faux Orphée : Apollon donna au Troyen Helenus le vrai\footnote{\Fontauri{C'est ainsi qu'il faut lire, et non Σιδηρίπς, qu'on lit dans le texte : le génitif Σιδηρῖταο qui se trouvé au ver 46e. détermine la vraie leçon.}} \emph{Sideritès}, que d'autres appellent \emph{Ophitès}, Pierre qui a le don de la parole : cette Pierre est un peu raboteuse, elle est dure, pesante et noire, et\footnote{\Fontauri{Αμφὶ δέ μιν κύκλω πρί τ᾽ ἀμφί τε πανθοτεν ἶνες Εῗκελοι ῥυτίδεαπ ἐπιχάβδην ταvύοvται. v. 20. et 21.}} a des rides qui s'étendent circulairement sur sa surface. Quand Helenus voulait employer la vertu de cette Pierre, il s'abstenait pendant 21. jours du lit conjugal, des bains publics, et de la viande des animaux ; ensuite il faisait plusieurs sacrifices, il lavait la Pierre dans une fontaine, l'enveloppait précieusement, et la portait dans son sein : après cette préparation qui rendait la Pierre animée, pour l'exciter à parler, il la prenait à la main, et faisait semblant de vouloir la jeter ; alors elle rendait un cri semblable à celui d'un enfant qui désire le lait de sa nourrice : Helenus profitant de ce moment interrogeait le Pierre sur ce qu'il voulait savoir, et en recevait des réponses certaines : c'est sur ces réponses qu'il prédit la ruine de Troy sa patrie.

J'ai honte de vous rapporter de pareilles fadaises ; quel autre nom donner à cela ? Mais Photius cet Écrivain grave et judicieux, n'a pas dédaigné de nous en conserver de la même force, dans son extrait de la Vie d'Isidore par Damascius. La ressemblance de ce que je trouvé touchant les \emph{Bætyles} dans cet extrait, avec ce que je viens de citer d'Orphée, m'engage à rapporter ici tout de suite ce que dit Photius ; quoique depuis Orphée, beaucoup d'autres Auteurs plus anciens que Photius, ayant fait mention de ces Pierres fabuleuses.

Photius\footnote{\Fontauri{Biblioth. cod. 242. p. 1047. 1062. et 1063. édit. A. Schotti.}} nous instruit donc avec un détail assez circonstancié, de ce que Damascius raconte des \emph{Bætyles}, et de leurs prodiges, dignes, ajoute Photius, de l'impiété de ces Philosophes païens ; disons aussi, dignes de leur extravagance. Ces Pierres qu'Isidore avait vues, aussi bien qu'Asclépiade et le médecin Eusèbe, ses amis, estoient d'une figure ronde, d'une grosseur médiocre, et avoient des lignes gravées sur leur surface. Damascius les appelle\footnote{\Fontauri{\emph{Ibid.} p. 1063. χάμματα ὀν τδ λίθω γεχαμμένα.}} \emph{lettres}, pour rendre la chose plus mystérieuse : effectivement ces lignes, que je crois estre précisément ce qu'Orphée appelle \emph{rides}, forment une apparence de caractères ; comme on peut l'observer par l'inspection même du \emph{Bætyle}. Le médecin Eusèbe avait un \emph{Bætyle}, qu'il portait quelquefois dans son sein ; d'autres fois il le plantait dans un trou de muraille : il l'interrogeait sur tour ce qu'il voulait savoir, et il en recevait des oracles\footnote{\Fontauri{\emph{Ibid.} p. 1063. Φωντιὺ ἀφίει λεπτοδ συρίσματοσ, τιὒ ἑρμηίευσεν ὁ Εὐσέβιος.}} d'une voix qui ressemblait à un petit sifflement, et qu'Eusèbe savait interpréter. Voilà les signes qui m'ont fait reconnaître les \emph{Bætyles} dans les \emph{Sideritès} d'Orphée. Ceux qui ont cru que le Poète continuait à parler de l'Aimant, ont este trompez par le mot \emph{Sideritis}, qui est un des noms que l'on a donnez à cette pierre, ainsi que je l'ai dit dans ma Dissertation sur l'Aimant.\footnote{\Fontauri{\emph{De Divinat. l. 3. c. 17.} où il rapporte très infidèlement le texte d'Orphée sans nommer cet Auteur, ajoutant un mot de celui de Jean Tzetzès, Chil. 6. Hist. 65. et non 57. qui n'est que l'extrait de celui d'Orphée.}} J. C. Boulanger est tombé dans cette erreur. Quant à\footnote{\Fontauri{Jean Frederic Herwart dan son livre intitulé \emph{Admiranda Ethnicæ Theologiæ mysteria}, Ingolstad. 1626. 4.$^\circ$}} Herwart qui ne balance pas à donner le \emph{Sideritès} d'Orphée pour l'Aimant, il lui suffisait de son propre fanatisme, à lui qui ne voyait qu'Aimant dans toute la nature, et qui ne regardait toutes les fables de la Mythologie, que comme autant de symboles, sous le voile desquels la boussole estoit déguisée.

Achevons ce qui regarde les autres singularités des \emph{Bætyles} rapportées par Damascius\footnote{\Fontauri{\emph{Photius, ibid.} p. 1047.}} : on les trouvait sur le Mont Liban ; ils y descendaient dans un globe de feu, et ils\footnote{\Fontauri{\emph{Ibid.} Διὰ τῧ ἀέρος κινόυμενοι.}} voltigeaient dans l'air. Ces deux circonstances ne doivent pas estre oubliées, non plus que la figure ronde et les lignes gravées sur leur surface : tous ces caractères nous seront utiles pour découvrir quelle espèce de corps sont les \emph{Bætyles} dans l'Histoire Naturelle. Du reste, il faut mettre avec le don de la parole, c'est-à-dire au rang du fabuleux, le mouvement spontanée qu'avait le \emph{Bætyle} du Médecin Eusèbe. Damascius\footnote{\Fontauri{\emph{Ibid.} p. 1063. où Photius dit τῶν δἐ Βαιτύλων ἂλλον ἂλλῳ αὶακεῖσθαι, et que Schott traduit mal, \emph{alium alii incumbere} : αὶακεῖσθαι est là peur αὶατεθεῖσθαι, ἀφιεροῦσθαι.}} remarque de plus, que chaque \emph{Bætyle} estoit consacré à une Divinité particulière, à laquelle, pour ainsi dire, il servait d'organe ; mais il apprit de son maître Isidore, que ces Pierres n'estoient animées que de certains génies mitoyens entre les bons et les mauvais : ce discernement est du moins aussi fin que celui du Philosophe Heraïscus,\footnote{\Fontauri{\emph{Photius, ibid.} p. 1050.}} qui distinguait par un certain mouvement intérieur, les Statués animées d'avec les inanimées.

Je ne peux m'empêcher de donner en passant le portrait du Philosophe Isidore, tel que nous le donné Damascius\footnote{\Fontauri{\emph{Ibid.} p. 1030.}} son disciple et son admirateur. Ce personnage original avait la face\footnote{\Fontauri{\emph{Ibid.} προσωπον τετράγωιον, τετράγωνον est là au propre : plus bas il est employé au figuré dans le portrait que Damascius fait d'Agapius, p. 1074. καὶ συλλήβδην εἰπεῖν ἐδόκει τετράγωνος εἶναι καὶ ηὖ τηὺ σαφἱαν. Suidas au mot τετράγωνος rapporte ce passage plus amplement que Photius, et rend ce mot par εὐσαθὴσ, ἑδραῖος.}} quarrée, les yeux\footnote{\Fontauri{\emph{Ibid.} Εστῶτας ἅμα Βεβαίουσ, ὺ ἐπιτρόχη κινουμένυς.}} fixes et en même temps roulants : un pareil visage, se récrie Damascius, estoit le portrait sacré de Mercure Dieu de l'Éloquence : des yeux si extraordinaires estoient le siège de Venus et de Minerve : l'esprit répondait au corps ; il trouvait dans ses songes la solution de toutes les questions qui l'embarrassaient ; aussi\footnote{\Fontauri{P. 1031. Ἔπ τηὺ ψυχὴν ἱπὸ τῆς ὀμφῆς κατεχόμενος.}} estoit-il soigneux de les raconter dès qu'il estoit éveillé : il savait voir les événements futurs dans un verre plein d'eau, et il avait appris d'une femme ce beau genre de divination. Ceux qui la pratiquent encore aujourd’hui, seront bien aises de savoir qu'ils y sont autorisez par l'usage qu'en faisait le bon Roy Numa.\footnote{\Fontauri{\emph{S. August. de Civit. Dei}, l. 7. c. 35. \emph{Boissard. de Divin.} c. 5. p. 17.}} Enfin pour achever de peindre Isidore, il estoit si familier avec les \emph{Bætyles}, qu'il connaissait parfaitement la qualité des Génies auxquels ils servaient de domiciles. Tel estoit ce Philosophe dont Damascius\footnote{\Fontauri{Selon l'extrait de Photius cod. 242. car le même Photius cod. 181. où il donné un extrait de la même vie bien plus abrégé, remarque que Damascius n'estimant que soi-même, ne donne à tous ceux qu'il loué, que des louanges malignes, et qu'en particulier il blâme encore plus Isidore qu'il ne le loue. A l'occasion de ces deux extraits du même livre que Photius donné en différents endroits, on peut observer que cet Auteur s'est fort peu soucié de garder aucun ordre dans sa Bibliothèque. Casaubon avait déjà remarqué ce défaut d'ordre, \emph{Casauboniana} p. 11. et 12.}} ne parle qu'avec une profonde vénération ; à qui Photius croit faire honneur en quelque manière, en le traitant seulement d'impie ; et qu'on jugerait digne aujourd’hui d'estre mis aux Petites-maisons. Mais il estoit d'une Secte\footnote{\Fontauri{Voyez les réflexions d'Holstenius sur cette Secte, p. 26. \emph{de Vitâ et Scriptis Porphyricis} dans le livre intitulé \emph{Epicleti Enchiridion}, \emph{Cebetis Tabula}, \emph{etc.} à \emph{Luca Holstenio Cantabrig.} 1655. 8.$^\circ$}} où il fallait estre fou de profession ; car j'apprends d'ailleurs, qu'il régentait dans l'École Athénienne\footnote{\Fontauri{\emph{Jonsius Histor. Philos.} lib. 3. c. 18.}} fondée par un Plutarque fils de Nestorius, auquel succédèrent Syrianus, Proclus, Marinus, tous Philosophes fameux, mais frappez au même coin que nôtre Isidore leur successeur, et entêtez d'un Pythagorisme réchauffé, mêlé de Chaldaïsme, et habillé à la Platonicienne ; sorte de Philosophie mise en crédit par Plotin, Porphyre et Iamblique, et dans les sources de laquelle nous trouverons l'origine du miraculeux ridicule des \emph{Bætyles}.

Isidore et Damascius vivaient vers le milieu du sixième siècle. Il est surprenant que dans un temps où la Religion Chrétienne avait fait de si grands progrès, et dans le centre de l'Empire de Justinien, il\footnote{\Fontauri{Peu avant le temps de Damascius, Ammonius fils d'Hermias, et son disciple l'Iatrosophiste Gésius, soutenaient publiquement à Alexandrie l'éternité du monde contre les Chrétiens. Voyez le Dialogue de Zacharie le Scholastique contre le sentiment de ces Philosophes.}} régnât encore parmi ce reste de Païens, des superstitions de la nature de celles dont nous venons de parler, et qu'elles fussent, pour ainsi dire, enseignées publiquement. Ceux qui prétendent que les Oracles cessèrent entièrement à la mort de J. C. s'ils ont connu les \emph{Bætyles}, n'ont pas crû sans doute que ces petits Oracles portatifs méritassent attention, ou, s'ils en avoient parlé, ils n'auraient pas manqué de dire que le Démon chassé des temples, et de ces lieux fameux qui lui servaient de théâtre, pour séduire des peuples entiers, s'estoit cantonné dans ces petites Pierres comme dans son dernier retranchement, pour entretenir dans l'idolâtrie, quelques particuliers tels que ces misérables Philosophes.

Revenons aux \emph{Bætyles} : leur règne a este fort long : nous venons de les voir en réputation dans le milieu du sixième siècle de l'Ere Chrétienne, et le prétendu Orphée les donné pour connus dès le temps de la guerre de Troy. Quoique cet Auteur qui voudrait se faire passer pour contemporain des derniers Troyens, n'ait\footnote{\Fontauri{\emph{Ger. Joan. Vossius de Poetis Græcis} c. 4.}} vécu que du temps de Pisistrate, c'en est assez pour regarder ce qu'il dit des \emph{Bætyles}, comme fondé sur une tradition déjà reçue dans la 65e. Olympiade : mais la connaissance de ces Pierres est presque aussi ancienne que le monde, si nous ajoutons foi au témoignage de Sanchoniathon, que Philon de Byblos son traducteur en Grec, nous donné pour un Auteur antérieur à la guerre de Troy. Il est vrai que quelques\footnote{\Fontauri{Joan. Henric. Ursinus et Dodwel ; pour M. Simon Bibl. Critiq. t. 1. c. 9. il attribue cette supposition à Porphyre ; mais il n'a pas pris garde qu'Athénée qui vivait plus de 50. ans avant Porphyre, avait fait mention des Φοινικικὰ de Sanchon. \emph{Deipn.} l. 3. c. 37. Le P. Calmet, Diction. de la Bible au mot \emph{Bethel}, a suivi le sentiment de M. Simon sans l'examiner.}} Savants révoquent en doute l'existence même de Sanchoniathon, et prétendent qu'il ne la doit qu'à Philon de Byblos, qui sous ce nom suppose son propre ouvrage : mais quand leurs raisons prévaudraient à celles d'un\footnote{\Fontauri{M. de la Croze, p. 174. de ses Entretiens sur divers sujets, \emph{etc.} \emph{Cologne} 1711. 12.$^\circ$ parle de cet Homme de lettres comme d'une tierce personne, quoiqu’il y ait apparence que ce soit lui-même.}} homme de lettres qui croit pouvoir défendre l'authenticité de Sanchoniathon, il suffirait de remarquer que Philon de Byblos auteur du second siècle, et Phénicien lui-même, en forgeant son Histoire sous le nom d'un des plus anciens de ses compatriotes, n'aurait pu donner quelque couleur à sa supposition pour l'accréditer, qu'en ramassant des traditions, soit historiques, soit mythologiques, connues d'ailleurs, et qui n'estoient point démenties par d'autres Historiens. Ainsi donc, ce que je vais citer du Sanchoniathon vrai ou faux, touchant les \emph{Bætyles}, doit estre regardé comme une opinion établie dans l'antiquité la plus reculée.

Eusèbe dans les fragments qu'il nous a conservez de cet Auteur Phénicien, dit\footnote{\Fontauri{\emph{Præparat. Evangel.} l. 1. c. 10. p. 37. \emph{édit. Viger.} Επενόκσε θεὸς Οὑρανὸς Βαιτύλια λίθους ἐμψύχους μηχαινσάμενος.}} que le Dieu Cœlus inventa les \emph{Bætyles} Pierres animées : le Dieu Cœlus, c'est-à-dire le Ciel : rien ne marque mieux l'origine de ces Pierres, qui, félon Damascius, descendaient de l'air dans un globe de feu. Eusèbe avait dit plus haut, que Bétul estoit un des quatre enfants de ce Dieu ; d'où il paraît vraisemblable que Cœlus aurait donné à ces Pierres le nom de son Fils, pour faire honneur à sa mémoire, ou par quelque autre raison que nous ignorons. Je ne fais ici cette réflexion, que par rapport à l'origine d'un mot que les Grecs revendiquent : mais nous ferons plus bas cette discussion.

Le reste des anciens Auteurs qui parlent du \emph{Bætyle}, comme Priscien le Grammairien, l'Auteur de l'Étymologicon et Hésychius, n'en donnent guéres d'autre notion, que comme de la pierre qu'avala Saturne. Hésychius\footnote{\Fontauri{Au mot Βαίτυλοσ, οὕτως ἐκαλοῖτο ὁ δοθεῖς λίθος τῳρ Κρόνῳ αἰπ᾽ Διός.}} n'en dit que cela précisément ; ce qui a donné occasion au proverbe contre les gens voraces : \emph{vous avaleriez même un Bætyle}.\footnote{\Fontauri{Καὶ Βαίτυλον αὒ καταπίνοισ, Erasme Adag. Chil. 4. centur. 2. Je ne çay d'où il a pris ce proverbe.}}

La Pierre que Saturne avala à la place de Jupiter, est fameuse dans la mythologie ; je ne m'y arrêterai que pour en observer quelques particularités moins connues. Agathocles\footnote{\Fontauri{Cité par le Scholiaste d'Hésiode sur le vers 485. de la Théogonie.}} le Babylonien dit que Rhéa prit cette Pierre dans l'Isle de Proconnése : Stephanus de Byzance\footnote{\Fontauri{Εθινικὰ au mot Θαυμάσιον ὄρος.}} rapporte que Saturne l'avala sur le Mont Thaumasius : Pausanias dans ses Arcadiques\footnote{\Fontauri{C. 36.}} dit la même chose d'après la tradition des gens du pays, et dans ses Bœotiques,\footnote{\Fontauri{C. 41. à la fin.}} que ce fut sur le Mont Petrachus. On lit dans les Dionysiaques de Nonnus,\footnote{\Fontauri{Lib. 25. vers la fin, φόρτον ἀποπτύων.}} que Saturne trouvant la pierre trop pesante, la vomit : Pausanias\footnote{\Fontauri{C. 24. vers la fin.}} avait déjà rapporté ce fait dans l'endroit de ses Phociques, où il dit qu'on voyait cette fameuse Pierre près du Temple de Delphes : c'estoit une Pierre de médiocre grosseur, qu'on arrosait tous les jours d'huile, et sur laquelle on mettait de la laine crue, les jours des feste.

Priscien et l'Auteur de l'Étymologicon, citez plus haut, disent, comme Hesychius, que la Pierre dont nous venons de parler, s'appelait \emph{Bætyle} ; mais ils ajoutent quelques circonstances qui méritent attention.

Priscien\footnote{\Fontauri{\emph{Grammatic}. l. 5.}} rend le mot \emph{Abbadir} par celui du \emph{Bætyle} qu'avala Saturne, après avoir dit immédiatement auparavant, qu'\emph{Abbadir} est un Dieu, ce que j'interprété du \emph{Bætyle} regardé comme animé d'une Divinité. Ce mot \emph{Abbadir} se trouvé corrompu dans les Gloses d'Isidore, où on lit \emph{Agadir lapis}, mot que Barthius\footnote{\Fontauri{\emph{Adversarior.} l. 16. c. 24. Reins. \emph{variar. lect.} l. 3. c. 17. remarque cette faute de Barthius ; mais il tombe lui-même dans quelques autres, que j'aurai peut-être ailleurs occasion de relever.}} prend tel qu'il est, et rapporte ridiculement à la langue Allemande. Bochart\footnote{\Fontauri{Chanaan l. 2. c. 2.}} avec raison cherche dans la langue Phénicienne l'origine d'\emph{Abbadir}, et croit avec vraisemblance, qu'il signifie une pierre ronde ; ce qui quadre avec la figure décrite par Damascius.

L'Auteur de l'Étymologicon après avoir dit, selon la restitution de Bochart,\footnote{\Fontauri{Chanaan l. 2. c. 2.}} \emph{Bætyle}, Pierre qui se trouvé sur le Mont Liban (circonstance conforme à ce que nous avons vu dans l'Extrait de Photius) ajoute qu'on appelle du nom de \emph{Bætyle}, la Pierre que Saturne avala, croyant avaler son fils Jupiter, et que le mot de \emph{Bætyle} vient du celui de la peau dans laquelle Rhéa avait emmailloté cette pierre en manière d'enfant. On contredirait\footnote{\Fontauri{Βαίτη, Βαίτυλος : comme στὀμα, στωμύλοσ, χρέμησ, χρεμύλοσ, \emph{etc.}}} vainement cette dérivation, sur un prétendu défaut d'analogie : la vraie et la seule raison contre l'origine Grecque, c'est que le mot est Phénicien ; Philon de Byblos l'a retenu dans son texte Grec, aussi bien que celui d'\emph{Il}, de \emph{Dagon}, et beaucoup d'autres : Bochart\footnote{\Fontauri{Chanaan l. 2. c. 2.}} aussi n'en a cherché l'étymologie que dans une langue Orientale : mais je conteste au même Bochart,\footnote{\Fontauri{\emph{Ibid.}}} une correction qu'il veut faire au passage de Sanchoniathon rapporté par Eusèbe, et que nous avons cité plus haut : il prétend que Philon de Byb. en traduisant de Sanchoniathon, \emph{Bætyles}, \emph{Pierres animées}, trompé par la ressemblance des lettres, a pris le mot qui dans la langue originale signifie \emph{animées}, pour celui que signifie \emph{ointes} ou \emph{graissées}. Le but de cette prétendue correction a este de rapporter toute la mythologie des \emph{Bætyles}, à la Pierre que Jacob arrosa d'huile : mais pour établir ce sentiment dont nous discuterons bientôt le fonds, est-il raisonnable de faire dire à Sanchoniathon, que Cœlus fabriqua des Pierres graissées d'huile, comme si cette onction pouvait entrer dans leur fabrication ? Nous avons déjà vu que le prétendu Orphée antérieur de beaucoup à Philon de Byblos, donné aux \emph{Bætyles}, l'épithété\footnote{\Fontauri{\emph{Titulo Ostrites} v. 28. λᾶαν... ἔμπνοον ἔρδεν.}} d'\emph{animez} ; ainsi il faudrait supposer qu'il eût connu le livre de Sanchoniathon, et qu'il eût fait la même bévue que Philon de Byblos, en traduisant le mot en question. On peut donc assurer que ce n'est pas l'épithété d'\emph{animées}, donnée par méprise à ces Pierres, qui a amené la ridicule propriété qu'on leur attribué ; c'est la propriété elle-même qui s'est établie sur une fausse opinion, dont la superstition, comme nous le ferons voir, a trouvé le fondement dans un fait fabuleux de l'Histoire Naturelle ; ensuite l'épithété a este donnée à la chose, conformément à l'opinion.

Bochart,\footnote{\Fontauri{Chanaan, l. 2. c. 2.}} après la correction dont je viens de parler, pour établir une parfaite conformité entre les \emph{Bætyles} et la Pierre de Jacob, n'oublie pas de rappeler l'étymologie du mot \emph{Bætyle} proposée\footnote{\Fontauri{V. ci-après la note 1. p. 523.}} déjà par plusieurs Savants, qui tirent ce mot de celui de \emph{Bethel}, nom\footnote{\Fontauri{Genèse c. 28. v. 12-19.}} que Jacob donna à l'endroit où il avait fait ce songe mystérieux dont l'Écriture parle, aussi bien qu'à la Pierre qui lui avait servi de chevet pendant son sommeil, et qu'à son réveil il arrosa d'huile. J'avoué que ce mot qui signifie \emph{Maison de Dieu}, convient parfaitement à l'idée qu'on avait des \emph{Bætyles} : mais cette dénomination même nous fournit un nouveau titre contre la correction de Bochart que j'ai combattue ; puisque rien n'est plus conséquent, que de donner l'épithété d'\emph{animée}, à une Pierre qui sert de domicile à une Divinité.

Ceux\footnote{\Fontauri{Comme Vandale, \emph{Dissertat. super Sanchoniat. ad calcem Dissert. super Aristea.}}} qui sont assez difficiles pour contester que le \emph{Béthel} de Jacob ait pu estre connu à Sanchoniathon, malgré le commerce que d'autres\footnote{\Fontauri{Comme M. Huet, \emph{Demonst. Evangel. Propos. 4.} c. 2-3.}} supposent que cet Historien a eu avec Gédéon, auront recours au Prince \emph{Bétul}, nom de même signification que \emph{Bethel}, et diront que Cœlus son père, ainsi que j'ai déjà observé, a nommé ces Pierres \emph{Bætyles} en l'honneur de son Fils. Quelque épaisses que soient les ténèbres, dont les faits de ces temps reculez sont enveloppez, il peut y avoir eu réellement un Prince \emph{Bétul}, comme l'on convient de l'existence de son frère Κρὁνοσ, ou Saturne appelé \emph{Il} dans ce même endroit de l'Historien Phénicien.

Hasarderai-je une conjecture, pour indiquer encore une autre origine ? les mots \emph{Abbadir} et \emph{Bætyle}, quelque différents qu'ils soient en apparence, ont cependant presque les mêmes lettres radicales : la fréquente permutation des Muettes Δ et T, et des Liquides Λ et P, les fait regarder comme les mêmes lettres dans toutes les langues.

J'en aurais trop dit sur le nom, si la chose même n'en dépendait. Bochart fondé sur son étymologie, aussi bien que sur sa prétendue correction, et la plupart\footnote{\Fontauri{Jos. Scaliger, G. J. Vossius, Th. Reines. H. Grotius, Selden, Huet, Heidegg. Witsius, \emph{etc.}}} des Savants avec lui, n'hésitent pas à prononcer que les Païens ont emprunté leurs \emph{Bætyles} du \emph{Béthel} de Jacob : mais quand la correction de Bochart ne serait pas vicieuse, et que j'admettrais l'étymologie, quelle conformité d'ailleurs pourrait-on trouver entre les \emph{Bætyles} et la Pierre du Patriarche ? elle me paraît à peu près la même que celle que certains\footnote{\Fontauri{\emph{Mœbius ex Taubmanno, in dedicat. libri sui de Oraculis.}}} Auteurs ont imaginée entre la chaîne d'or décrite par Homère, et l'échelle que Jacob vit en songe. 

La Pierre de Jacob devait estre d'une grosseur assez considérable et d'une figure à peu près quarrée ; puisqu'il la dressa\footnote{\Fontauri{\emph{Erexit in titulum. Genes.} c. 28. v. 18.}} en forme de colonne : elle estoit par conséquent immobile, et ne pouvait\footnote{\Fontauri{En effet cette Pierre élevée devint un autel, selon S. Augustin \emph{de Civit. Dei} lib. 16. c. 38.}} avoir d'autre usage que celui d'un autel. Les \emph{Bætyles} au contraire, estoient d'une grosseur très médiocre ; leur figure estoit ronde, et ils estoient portatifs ; de plus ils avoient des cannelures gravées sur leur surface, et selon l'opinion commune, ils descendaient du Ciel : circonstances qu'il ne faut point perdre du vue, et par où les \emph{Bætyles} sont caractérisez de manière à n'avoir rien de commun, non seulement avec la Pierre de Jacob, mais encore avec toutes les autres Pierres\footnote{\Fontauri{Ainsi Marsham p. 55. édit. \emph{Fol.} se trompe en disant que Sanchoniathon appelle Βαιτύλια, les premières Statués qui furent adorées. Le même Marsham suit en tout, le sentiment de Bochart.}} qui servaient au culte des Païens. Ce serait parmi ces dernières, celles\footnote{\Fontauri{Voyez-en les exemples ramassez dans Duportus \emph{in Theoph. charact.} sur ces mots du c. 16. τῶν λιπαρωρν λίθων. Dans Dought. \emph{Analect. sacr. Excurs. 17. in Genesin} ; et dans Broukhus. \emph{in Tibull.} l. 1. \emph{Eleg.} 1. v. 15-16.}} qu'on trouvé si fréquemment dans les Auteurs Sacrez et Profanes, arrosées d'huile, et couronnées de fleurs, qui auraient quelque ressemblance avec la Pierre de Jacob. Scaliger\footnote{\Fontauri{\emph{Animadvers. in Chronol. Euseb.} p. 216. et d'après lui Bochart qui nous rapporte le texte Hébreu, et y ajoute la traduction. \emph{Chan.} l. 2. c. 2.}} sur Eusèbe, nous rapporte un passage Hébreu, où il est dit que Dieu prit en haine la Pierre de Jacob ; parce que les Chananéens en firent l'objet de leur idolâtrie : c'est sans doute ce qui\footnote{\Fontauri{Jos. Scaliger \emph{Not. in fragment. Veter. Græc.} p. 24 \emph{ad calcem operis de Emend. Temp.} et G. J. Voss. \emph{de Theolog. Gentil.} l. 6. c. 39. Le P. Calmet dans son Dicton. de la Bible, rapporte un peu autrement la cause de ce changement de nom. Voyez Hadr. Reland dans sa Palestine illustrée, T. 2. aux mots \emph{Bethaven} et \emph{Bethel}.}} fit changer le nom du lieu appelé \emph{BETHEL}, \emph{Maison de Dieu}, en celui de \emph{BETH-AWEN}, \emph{Maison du mensonge}. On voit là simplement que les Chananéens faisaient de la Pierre de Jacob, le même usage que faisait en général de semblables Pierres, le commun des Païens. On peut même tirer de-là l'origine de la superstition des Juifs à l'égard du λίθος σκοπὸς dont il est parlé dans le Lévitique,\footnote{\Fontauri{C. 26. v. 1.}} où les Idoles et les Pierres dressées, leur sont également défendues : mais est-il bien sûr que l'idolâtrie des autres Nations ait commencé par cette perverse imitation des Chananéens ? Avant Jacob il y avait eu des Idolâtres et des Pierres, aussi bien que d'autres corps, soit animez, soit inanimés, qui faisaient l'objet du culte des Chaldéens. Ce serait donc chez ces peuples qu'il faudrait chercher l'origine de l'idolâtrie des autres Nations ; puisque c'est de leur sein qu'elle se répandit sur toute la terre, avant la vocation d'Abraham, et bien longtemps avant la Loy de Moyse.

J'admire les travaux de ces grands hommes\footnote{\Fontauri{Selden, Bochart, Huet, Prideaux, \emph{etc.}}} qui ont employé la plus vaste érudition, à rechercher les sources de la mythologie et des superstitions du Paganisme, dans l'histoire et dans la Religion des Juifs : souvent même c'est encore plus par une sagacité singulière, que par une lecture infinie et par une profonde connaissance des langues, qu'ils entrevoient des rapports dont les hommes ordinaires sont étonnez. Je çay que les premiers\footnote{\Fontauri{Justin, Clément Alexandrin, Tertullien, Eusèbe, \emph{etc.}}} Pères de l'Eglise se crurent obligez de faire valoir ces mêmes idées, contre les Païens dont ils estoient environnez, dans le dessein de mortifier leur orgueil, et de leur faire reconnaître la corruption de leur culte en leur en découvrant l'origine ; mais je ne çay si la doctrine qu'ils prêchaient, et l'exemple des vertus qu'ils pratiquaient, n'ont pas este plus efficaces que cette manière de raisonner, contre des Philosophes aussi instruits que ceux qu'ils avoient à combattre. Aujourd’hui de pareils secours sont encore moins utiles à la Religion : il s'est élevé dans ce siècle un genre de Philosophes plus dangereux que les anciens : en hasardant des preuves si douteuses, et d'ailleurs si peu importantes, on doit craindre de fournir à ces nouveaux Philosophes, le léger avantage qu'on croit procurer à la bonne cause, qui en a de si grands et de si incontestables par les endroits essentiels. L'Esprit Philosophique qui a travaillé si heureusement pour mettre la vérité de nôtre Religion dans tout son jour, quand il ne trouvera pas un fondement réel à ce rapport du culte des Païens avec celui des Juifs, expliquera aisément ce phénomène de ressemblance, sans allusion forcée, en le rappelant à sa propre cause : c'est l'uniformité de l'homme avec lui-même, dans les pays et dans les temps les plus éloignez, sur l'idée d'une Divinité, et sur les points les plus généraux d'un culte que tous croient également lui devoir. \emph{Adeo\footnote{\Fontauri{C'est ce que Pline dit de la Magie, à la fin du c. 1. du 30e. l. de son Histoire : ce qu'on peut dire de la Religion avec bien plus de raison.}} ista toto mundo consensere, quanquam discordi et sibi ignoto}.

Il me reste une mention des \emph{Bætyles}, dans un Auteur peu connu, parce qu'il est demeuré Manuscrit : c'est un Joseph Chrétien\footnote{\Fontauri{V. G. Cave \emph{Scriptor. Ecclesiast. Histor. Sæcul. Nestorian.}}} du 5e. siècle, différent de Joseph de Tibériade : plusieurs Savants citent son \emph{Hypomnesticum} : Thomas Gale, qui en avait une copie tirée sur le MS. de la Bibliothèque de Cambridge, en rapporte un passage assez long dans ses Notes\footnote{\Fontauri{P. 215.}} sur le livre des Mystères de Iamblique. Dans ce passage, Joseph après avoir parlé de plusieurs espèces de divination, ajoute : τὰ ἐν τοῖς ναοῖς Βαιτύλια δζὰ λίθιον ἐν τοῖς στοιχείοις προσρασθόντων. Je ne çay quel sens donner à ces mots, si l'on n'y fait quelque correction : je croirais qu'il faut lire δζὰ λίθιον ἐν τοῖς τοίχοις προσχρνσαύ των ; \emph{Bætyles des temples, genre de divination, qui se fait par le moyen de certains Pierres enchâssées dans les murailles, et qui de-là rendent des Oracles}. Nous avons vu dans l'extrait de Photius, que le Médecin Eusèbe plantait son \emph{Bætyle} dans le mur, quand il voulait l'interroger ἐν τοῖχω ἐγκρόυσας.\footnote{\Fontauri{\emph{Phot. Biblioth.} p. 1063.}} Si l'on veut conserver dans ce passage, les mots ἐν τοῖς στιχέιοισ, cela signifiera que ces Pierres rendaient des Oracles par la vertu des lettres gravées sur leur surface : Damascius,\footnote{\Fontauri{\emph{Ibid.}}} ainsi que nous avons remarqué, appelait \emph{lettres}, ces lignes raboteuses qu'Orphée décrit sous le nom de \emph{rides} ; mais pour parler plus sûrement, il faudrait voir le MS. Il n'y a à la Bibliothèque du Roy, qu'un fragment de l'ouvrage de ce Joseph, et dans ce fragment, le passage dont il est question, ne se trouvé point.

Outre les Pierres appelées expressément \emph{Bætyles}, ou caractérisées de la manière que nous avons observée, on en trouvé dans les Auteurs, quelques autres qui, sans estre nommées, pourraient estre soupçonnées de la même espèce. Les Pierres, par exemple, qu'Élagabal transporta à Rome, appelées par Lampridius, \emph{Lapides Divi}, sont regardées par Saumaise\footnote{\Fontauri{\emph{Not. in Lamprid.} p. 181.}} comme des \emph{Bætyles} ; mais ces \emph{Bætyles} n'ont d'autre titre que la correction de ce Critique qui veut qu'on lise \emph{vivi}, au lieu de \emph{Divi}. Saumaise d'ailleurs n'entend pas bien ce passage, pour n'avoir pas pris garde qu'il y manque un mot nécessaire : Tristan\footnote{\Fontauri{Comment. Historiq. t. 2. p. 324.}} supplée ce mot, et se trompe\footnote{\Fontauri{V. les notes de G. Cuper \emph{in Lactant. de mortibus Persecut.} p. 156-158.}} en même temps, lorsqu'il prétend que les Pierres en question soient celles qui estoient à Phares ville d'Achaïe, près de la Statué de Mercure : mais j'abandonné les \emph{Lapides Divi} de Lampridius, à des conjectures plus heureuses.

Je serais mieux fondé à faire passer pour \emph{Bætyles}, les Pierres\footnote{\Fontauri{\emph{Plutarch. de Fluviis. S. Eurotas.}}} que l'on consacrait dans le Temple de Minerve Chalcidique à Sparte ; elles en avoient du moins la figure et le mouvement. Le Plutarque auteur du livre des Fleuves, dit qu'on les prenait dans le fleuve Eurotas ; que leur figure ressemblait à celle d'un casque ; qu'au son de la trompette elles s'élevaient sur l'eau ; et qu'au nom des Athéniens, sitôt qu'il estoit prononcé, elles se replongeaient au fond du fleuve : circonstances d'où elles avoient reçu le nom de Θρασύδειλοι. La fable de ce mouvement ridicule est manifestement tirée de l'aventure du Prince Eurotas, dont il est parlé au même endroit, et elle n'a rien de commun avec celle du mouvement spontanée des \emph{Bætyles} ; mais cette figure de casque leur convient parfaitement, ainsi qu'on peut le reconnaître par l'inspection même de la Pierre, que je prouverai avoir este le \emph{Bætyle} des Anciens.

Il y a certaines autres Pierres célèbres dans la Mythologie, qui, bien qu'elles ne soient pas de vrais \emph{Bætyles}, doivent selon moi en estre regardez comme des espèces, par rapport à leur origine commune : ce sont les Pierres tombées du Ciel ; je ne veux pas dire les pluies de pierres si souvent rapportées parmi les prodiges : j'entends parler uniquement de ces Pierres singulières qu'on croyait envoyées du Ciel par quelque Divinité, qui voulait se manifester, et estre adorée sous cette figure.

Telle estoit la Pierre décrite par Hérodien,\footnote{\Fontauri{\emph{Historiar.} lib. 3. c. 3.}} adorée à Emèse, comme représentant le Soleil, dont Elagabal, dans sa jeunesse, estoit le Prestre : la Pierre se voit dans plusieurs\footnote{\Fontauri{Une des Ephésiens, Vaillant \emph{Numism. Imperat. à Pop. Græc. etc. Edit. altera.} p. 127. deux autres, Vaillant \emph{Numism. Imperat. præstant. Edit. altera.} t. 2. p. 285-288.}} médailles de cet Empereur. On estoit déjà accoutumé à adorer le Soleil sous cette figure : les Pierres\footnote{\Fontauri{\emph{Natural. Histor.} l. 2. c. 58 et \emph{Plutarch. in Lysandri vita.}}} tombées de cet Astre selon la prédiction d'Anaxagore, avoient reçu les honneurs divins à Abyde et à Potidée. Je ferai voir plus bas que parmi ces Pierres on trouvé une espèce de \emph{Bætyle}.

La Pierre de Venus Paphiene, estoit à peu près de la même figure que celle du Soleil à Emèse : elle est représentée\footnote{\Fontauri{Sur une d'Eurypile, Spanh. \emph{de præstant. Numism. Dissert.} 8e. t. 1. p. 505. De Drusus, Patin \emph{Imperat. Rom. Numism. ex ære mediæ et minim formæ}, p. 80. De Trajan, Tristan \emph{Comment. Histor.} t. 1. p. 419. De Caracalla, \emph{Id. ibid.} t. 2. p. 220.}} aussi sur plusieurs médailles : les Auteurs\footnote{\Fontauri{\emph{Tacit. Histor.} lib. 2. c. 2. et \emph{Maxim. Tyr. Dissert.} 38.}} qui en parlent comme d'une Pierre d'une espèce inconnue, ne disent pourtant pas qu'elle fût tombée du Ciel ; mais sa figure pyramidale comme celle du Soleil, me la fait croire de la même nature, aussi bien que la Pierre d'Apollon Carinus,\footnote{\Fontauri{\emph{Pausan. Attic.} lib. 1. c. 44.}} celle de Jupiter Milichius,\footnote{\Fontauri{\emph{Idem Corinth.} lib. 2. c. 9.}} et peut-être beaucoup d'autres dont je prouverai dans la suite, l'affinité avec les \emph{Bætyles}, selon les opinions reçues par les Anciens, dans l'Histoire Naturelle.

La Pierre de la Mère des Dieux estoit d'une espèce singulière, et paraît n'avoir aucun rapport avec celles dont je viens de parler ; mais elle\footnote{\Fontauri{\emph{Herodian. Histor.} lib. 1. c. 11. et \emph{Appian. Annibalic.}}} estoit tombée du Ciel ; elle estoit\footnote{\Fontauri{\emph{Tit. Liv.} lib. 29. S. 14.}} d'une grandeur médiocre, puisqu'elle se portait aisément à la main ; sa couleur\footnote{\Fontauri{\emph{Arnob. advers. Gent.} lib. 7. et \emph{Prudent. Hymn. in Roman.}}} estoit noire, sa figure, quo qu’irrégulière, avait quelque chose de symétrisé : au milieu de toutes ses inégalités, on trouvait une apparence de bouche ; ce qui donna l'idée d'enchâsser la Pierre à l'endroit de la bouche, dans le visage de la Statué de la Déesse. Je ne croirais pas estre téméraire en assurant que nous avons encore aujourd’hui la Pierre de la Mère des Dieux, dans ces Pierres figurées, que les Naturalistes appellent \emph{Hystérolithes} : peut-être même que par rapport à une ressemblance qui n'est guère éloignée de celle de la bouche, le culte de la Pierre fut imaginé : et on ne crut point trouver de symbole plus convenable que cette Pierre ainsi figurée, pour représenter une Déesse, qui selon les Poètes estoit la Mère des Dieux et des hommes, et qui selon les Philosophes, estoit la Nature même, source seconde de tout ce qui paraît dan l'Univers.

Je sens que je m'écarte beaucoup ; mais je ne saurais encore venir au point capital de ma Dissertation, sans avoir examiné le sentiment de ceux qui prétendent que le \emph{Jupiter Lapis} employé dans la formule d'un serment ordinaire aux Romains, est le \emph{Bætyle} qu'avala Saturne. Cicéron,\footnote{\Fontauri{\emph{L. 7. Epistol. 12. ad Trebat.}}} A. Gelle,\footnote{\Fontauri{\emph{Noct. Atticar.} lib. 1. c. 21.}} Apulée\footnote{\Fontauri{\emph{De Deo Socratis.}}} se servent de la formule, \emph{Jovem lapidem jurare} ; mais Festus\footnote{\Fontauri{\emph{De Verbor. Signific. Lapidem, Silicem, etc.}}} nous rend raison de cette expression, \emph{Lapidem, silicem tenebant juraturi per Jovem, hæc verba dicentes} : \emph{si sciens fallo, tum me Diespiter salva urbe arceque ejiciat, ut ego hunc lapidem} : d'où l'on doit conclure que \emph{Jovem lapidem jurare} est une locution elliptique, et qu'après le mot \emph{lapidem} il faut sous-entendre le participe \emph{tenens}. Les exemples détaillez qu'on trouvé dans Polybe\footnote{\Fontauri{\emph{Histor.} lib. 3. parlant du 3e. traité des Rom. et des Carthaginois.}} et dans Plutarque\footnote{\Fontauri{Dans lavie de Sylla, parlant du serment que Marius fit faire à Cinna.}} de la manière dont se faisait ce serment, ne permettent pas d'en douter.

Un Auteur\footnote{\Fontauri{\emph{Sam. Puiscus Lexicon Antiq. Roman.}}} de réputation qui a donné depuis peu un Dictionnaire d'antiquités, pour autoriser le prétendu \emph{Jupiter Lapis}, nous rapporte de la Chronique d'Eusèbe\footnote{\Fontauri{\emph{Anno à nato Abrah.} 554. selon l'édition de Scaliger.}} un \emph{Lapis} qui a régné en Crète : s'il avait pris la peine de voir le texte Grec, il aurait trouvé Λάπις, mot qui certainement n'a pas este fait sur le Latin \emph{lapis} ; puisque ce Λάπις régnait en Crète longtemps avant la guerre de Troy. On aurait tort de souhaiter que cet Auteur eût compris dans son livre, les Antiquités Grecques ; à juger par cet échantillon, nous devons lui estre obligez de n'avoir pas fait cette entreprise.

Il faut pourtant avouer qu'il y avait chez les Grecs, un \emph{Jupiter lapideus}, λιθίοτος ; mais il n'a pas plus de rapport aux \emph{Bætyles}, que le Pierre auprès de laquelle, et non par laquelle juraient, selon le décret de Solon, les Magistrats d'Athènes,\footnote{\Fontauri{5. \emph{Meursius Atticar. lection.} lib. 1. c. 6.}} appelez par cette raison λιθόμοται. Cependant un Commentateur\footnote{\Fontauri{\emph{Thysius in A. Gell.} l. 1. c. 21.}} sur A. Gelle, copié ensuite par d'autres,\footnote{\Fontauri{\emph{Servat. Gallæus in Oraculis Sibyll.} p. 486.}} nous entasse ces , le \emph{Jupiter lapis}, et les \emph{Bætyles}. Ces fautes de jugement ordinaires aux Commentateurs de profession, servent à mieux faire sentir le mérite de celui qui uniquement attaché à éclaircir son texte, sait dispenser avec sobriété et avec discernement, les trésors de son érudition.

Revenons aux \emph{Bætyles}, pour ne les plus quitter. Nous avons vu ceux qui estoient expressément ainsi nommez ; nous avons ensuite parcouru les Pierres qui pouvaient y avoir quelque rapport vrai ou faux : il en reste encore quelques-unes qui sans porter le nom de \emph{Bætyles}, non seulement sont de vrais \emph{Bætyles}, mais serviront à nous dévoiler la nature de ces Pierres. Sanchoniathon cité par Eusèbe\footnote{\Fontauri{\emph{Chanaan} lib. 2. c. 2.}} dans le même endroit où il est parlé des \emph{Bætyles}, dit qu'Astarte trouva une étoile tombée de l'air, et que l'ayant ramassée, elle la consacra dans la ville de Tyr : c'est la traduction littérale du Grec, εἷρεν ἀεροπετῆ ἀστέρα ὃν καὶ αὐελομένη ἐν Τύρω τῆ ἁγία νήσω ἀφιέρωσε. Bochart qui traduit, ὃν αὐελομένὴ, \emph{quam intersectam}, trouvé ridicule de faire tuer une étoile : le ridicule est dans sa traduction ; comme si αὐαιρεῖσθαι ne signifiait pas \emph{auferre}, aussi bien qu'\emph{interficere}, et que ce ne fût pas même là sa signification primitive : mais il avait envie de faire de cette étoile, une espèce d'Aigle appelée ἀστερίας, fondé sur une histoire rapportée par Nonnus.

Je vous paraîtrai sans doute encore moins raisonnable que Bochart, quand j'assurerai qu'il n'y a rien à changer au mot ἀστέρα, et que cette étoile prise à la lettre est une vraie Pierre du genre des \emph{Bætyles} : vous me trouverez tout aussi extraordinaire, quand je vous donnerai pour une étoile, la Pierre qui, selon la prédiction d'Anaxagore, tomba à Ægospotamos ville de la Chersonèse Taurique : mais j'espère en vous exposant le système que nous allons développer, vous prouver évidemment cet étrange paradoxe, que certaines étoiles en l'air, estoient des \emph{Bætyles} allumez, et que ces \emph{Bætyles} sur la terre, estoient des étoiles éteintes.

Pline\footnote{\Fontauri{\emph{Natur. Histor.} lib. 37. c. 9.}} nous parle de la Pierre \emph{Astroitès}, dont il dit que Zoroastre célébré les grandes vertus pour les opérations magiques. Cet \emph{Astroitès}, selon moi, se trouvé sous le nom de \emph{Pierre} simplement, sans autre addition, dans ce qui nous reste des Oracles supposez de Zoroastre : c'est précisément\footnote{\Fontauri{Ἡνἰκα δ᾽ ἐρχόμενον πρόσγειον δαἱμον᾽ ἀθρήσης Θῦε λίθον, Μνίζουρα ἐπαυδὢν.}} à la fin de ces fragments, qu'il est recommandé d'offrir en sacrifice une \emph{Pierre}, lorsqu'on verra un Démon terrestre s'approcher : en éclaircissant ce point de Théurgie, qui est important pour nôtre sujet, il paraîtra très vraisemblable que cette Pierre estoit un \emph{Bætyle}. Enfin Porphyre\footnote{\Fontauri{S. 17. Ὲκαθάρθη τῇ Κεραυνία λίθω.}} dans la Vie de Pythagore, dit que ce Philosophe estant arrivé en Crète, se fit purifier avec la \emph{Pierre de foudre}, par le Prestre Morgus un des Dactyles Idéens : voilà un nom qui commence à dévoiler la nature des \emph{Bætyles} : et je vais démontrer que le \emph{Bætyle} est une espèce de \emph{Pierre de foudre} : la preuve est évidente ; je la tire de Pline dont le témoignage est formel : il est étonnant qu'aucun de ceux qui ont parlé des \emph{Bætyles}, n'ait su prendre la vraie notion de ces Pierres dans le passage suivant de ce Naturaliste : \emph{Sotacus et alia duo genera fecit Cerauniæ, nigras rubentesque, ac similes eas esse securibus ; per illas quæ nigræ sunt et rotundæ, urbes expugnari et classes, easuqe BETULOS vocari ; qua vero longæ sunt, Ceraunias.}\footnote{\Fontauri{\emph{Natural. Histor.} lib. 37. c. 9.}}

Voici de nouvelles fables, mais qui coulent d'une autre source : c'est une Philosophie, la plus ancienne peut-être qu'il y ait au monde, laquelle semble n'avoir entrepris l'explication des Causes Naturelles, que pour nourrir la superstition des premiers hommes : mais la matière de la discussion, où je vais m'engager, est si abondante, que je crois devoir la réserver pour une seconde partie.
\clearpage
\end{document}
